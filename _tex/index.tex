% Options for packages loaded elsewhere
\PassOptionsToPackage{unicode}{hyperref}
\PassOptionsToPackage{hyphens}{url}
\PassOptionsToPackage{dvipsnames,svgnames,x11names}{xcolor}
%
\documentclass[
]{agujournal2019}

\usepackage{amsmath,amssymb}
\usepackage{iftex}
\ifPDFTeX
  \usepackage[T1]{fontenc}
  \usepackage[utf8]{inputenc}
  \usepackage{textcomp} % provide euro and other symbols
\else % if luatex or xetex
  \usepackage{unicode-math}
  \defaultfontfeatures{Scale=MatchLowercase}
  \defaultfontfeatures[\rmfamily]{Ligatures=TeX,Scale=1}
\fi
\usepackage{lmodern}
\ifPDFTeX\else  
    % xetex/luatex font selection
\fi
% Use upquote if available, for straight quotes in verbatim environments
\IfFileExists{upquote.sty}{\usepackage{upquote}}{}
\IfFileExists{microtype.sty}{% use microtype if available
  \usepackage[]{microtype}
  \UseMicrotypeSet[protrusion]{basicmath} % disable protrusion for tt fonts
}{}
\makeatletter
\@ifundefined{KOMAClassName}{% if non-KOMA class
  \IfFileExists{parskip.sty}{%
    \usepackage{parskip}
  }{% else
    \setlength{\parindent}{0pt}
    \setlength{\parskip}{6pt plus 2pt minus 1pt}}
}{% if KOMA class
  \KOMAoptions{parskip=half}}
\makeatother
\usepackage{xcolor}
\setlength{\emergencystretch}{3em} % prevent overfull lines
\setcounter{secnumdepth}{5}
% Make \paragraph and \subparagraph free-standing
\makeatletter
\ifx\paragraph\undefined\else
  \let\oldparagraph\paragraph
  \renewcommand{\paragraph}{
    \@ifstar
      \xxxParagraphStar
      \xxxParagraphNoStar
  }
  \newcommand{\xxxParagraphStar}[1]{\oldparagraph*{#1}\mbox{}}
  \newcommand{\xxxParagraphNoStar}[1]{\oldparagraph{#1}\mbox{}}
\fi
\ifx\subparagraph\undefined\else
  \let\oldsubparagraph\subparagraph
  \renewcommand{\subparagraph}{
    \@ifstar
      \xxxSubParagraphStar
      \xxxSubParagraphNoStar
  }
  \newcommand{\xxxSubParagraphStar}[1]{\oldsubparagraph*{#1}\mbox{}}
  \newcommand{\xxxSubParagraphNoStar}[1]{\oldsubparagraph{#1}\mbox{}}
\fi
\makeatother


\providecommand{\tightlist}{%
  \setlength{\itemsep}{0pt}\setlength{\parskip}{0pt}}\usepackage{longtable,booktabs,array}
\usepackage{calc} % for calculating minipage widths
% Correct order of tables after \paragraph or \subparagraph
\usepackage{etoolbox}
\makeatletter
\patchcmd\longtable{\par}{\if@noskipsec\mbox{}\fi\par}{}{}
\makeatother
% Allow footnotes in longtable head/foot
\IfFileExists{footnotehyper.sty}{\usepackage{footnotehyper}}{\usepackage{footnote}}
\makesavenoteenv{longtable}
\usepackage{graphicx}
\makeatletter
\newsavebox\pandoc@box
\newcommand*\pandocbounded[1]{% scales image to fit in text height/width
  \sbox\pandoc@box{#1}%
  \Gscale@div\@tempa{\textheight}{\dimexpr\ht\pandoc@box+\dp\pandoc@box\relax}%
  \Gscale@div\@tempb{\linewidth}{\wd\pandoc@box}%
  \ifdim\@tempb\p@<\@tempa\p@\let\@tempa\@tempb\fi% select the smaller of both
  \ifdim\@tempa\p@<\p@\scalebox{\@tempa}{\usebox\pandoc@box}%
  \else\usebox{\pandoc@box}%
  \fi%
}
% Set default figure placement to htbp
\def\fps@figure{htbp}
\makeatother

\usepackage{url} %this package should fix any errors with URLs in refs.
\usepackage{lineno}
\usepackage[inline]{trackchanges} %for better track changes. finalnew option will compile document with changes incorporated.
\usepackage{soul}
\linenumbers
\makeatletter
\@ifpackageloaded{caption}{}{\usepackage{caption}}
\AtBeginDocument{%
\ifdefined\contentsname
  \renewcommand*\contentsname{Table of contents}
\else
  \newcommand\contentsname{Table of contents}
\fi
\ifdefined\listfigurename
  \renewcommand*\listfigurename{List of Figures}
\else
  \newcommand\listfigurename{List of Figures}
\fi
\ifdefined\listtablename
  \renewcommand*\listtablename{List of Tables}
\else
  \newcommand\listtablename{List of Tables}
\fi
\ifdefined\figurename
  \renewcommand*\figurename{Figure}
\else
  \newcommand\figurename{Figure}
\fi
\ifdefined\tablename
  \renewcommand*\tablename{Table}
\else
  \newcommand\tablename{Table}
\fi
}
\@ifpackageloaded{float}{}{\usepackage{float}}
\floatstyle{ruled}
\@ifundefined{c@chapter}{\newfloat{codelisting}{h}{lop}}{\newfloat{codelisting}{h}{lop}[chapter]}
\floatname{codelisting}{Listing}
\newcommand*\listoflistings{\listof{codelisting}{List of Listings}}
\makeatother
\makeatletter
\makeatother
\makeatletter
\@ifpackageloaded{caption}{}{\usepackage{caption}}
\@ifpackageloaded{subcaption}{}{\usepackage{subcaption}}
\makeatother

\usepackage{bookmark}

\IfFileExists{xurl.sty}{\usepackage{xurl}}{} % add URL line breaks if available
\urlstyle{same} % disable monospaced font for URLs
\hypersetup{
  pdftitle={Яка машиночитаність потрібна для доступності використання відкритих даних},
  pdfauthor={Микола Кузін},
  pdfkeywords={машиночитаність, відкриті дані},
  colorlinks=true,
  linkcolor={blue},
  filecolor={Maroon},
  citecolor={Blue},
  urlcolor={Blue},
  pdfcreator={LaTeX via pandoc}}



\draftfalse

\begin{document}
\title{Яка машиночитаність потрібна для доступності використання
відкритих даних}

\authors{Микола Кузін\affil{1}}
\affiliation{1}{BRDO, }
\correspondingauthor{Микола Кузін}{m.kuzin@brdo.com.ua}


\begin{abstract}
Машиночитаність --- широкий термін, що може позначати як технічні вимоги
до оформлення веб-ресурсів, що уможливлює їхню автоматизовану взаємодію
в мережі Інтернет (для прикладу, концепція 5-Star Linked Data), так і
доступність (легкість) автоматизованої обробки з боку живих
користувачів. За кожним з цих значень стоять різні стандарти та практики
--- і ці значення слід розрізняти між собою, готуючи набори відкритих
даних до публікації.
\end{abstract}





\section{Вільний доступ vs доступність використання відкритих
даних}\label{ux432ux456ux43bux44cux43dux438ux439-ux434ux43eux441ux442ux443ux43f-vs-ux434ux43eux441ux442ux443ux43fux43dux456ux441ux442ux44c-ux432ux438ux43aux43eux440ux438ux441ux442ux430ux43dux43dux44f-ux432ux456ux434ux43aux440ux438ux442ux438ux445-ux434ux430ux43dux438ux445}

За визначенням Open Knowledge Foundation ---розробників CKAN, на якій
реалізований український Портал відкритих даних --- ``відкритість''
даних полягає в тому, що будь-хто може мати до них вільний доступ,
вільно використовувати, змінювати та ділитися ними\footnote{\url{https://opendefinition.org/od/2.0/ua/}}.

\href{https://aims.gitbook.io/open-data-mooc/unit-1-open-data-principles/lesson-1.1-what-is-open-data\#id-5.-challenges}{Однак
``вільний'' не завжди значить ``відкритий''}: часто потрібні додаткові
кроки, щоби з інформації у вільному доступі зробити дані, доступні до
використання.

Доступність використання --- важлива категорія. У Постанові КМУ № 835
принцип ``доступності використання'' відкритих даних напряму
пов'язується з машиночитаним форматом оприлюднених даних. А
машиночитаність --- зі структурованістю, що уможливлює обробку без
участі людини (власне, машинну обробку). У тому ж документі визначено
перелік структурованих форматів (розширень), що використовуються для
оприлюднення наборів. Сюди відносяться .json, .csv, .xls(x) та інші.

Але візьмемо набір даних
``\href{https://data.gov.ua/dataset/8b9b1677-2407-454a-bfa7-76eb638c0ea1}{Єдиний
звіт про кримінальні правопорушення}'', що оприлюднюється Генеральною
Прокуратурою на Порталі відкритих даних. Цей набір даних оприлюднюється
у структурованому форматі .xlsx і відтак, відповідно до визначень вище,
відповідає принципу доступності використання. Втім, заглянуваши
всередину опублікованих ресурсів (файлів) набору, побачимо, чому про
``машиночитаність'' та ``доступність використання'' тут можна говорити
досить умовно:

\begin{itemize}
\tightlist
\item
  Всередині однієї вкладки можуть розміщуватися кілька таблиць, одна під
  одною. Розрізнити їх між собою для роботи в середовищі розробки ``без
  участі людини'' складно.
\item
  Самі таблиці використовуються для відображення ієрархічних структур
  даних: для прикладу, за рядком ``Усього кримінальних правопорушень''
  слідують рядки з різнорівневими значеннями ``з них''. Екселівські
  таблиці --- не кращий спосіб відображення даних, що мають ієрархічну
  структуру.
\item
  Назви колонок йдуть у кілька рядків, частина цих клітинок злита між
  собою, що далі ускладнює зчитування цих таблиць у середовище розробки.
\item
  Значення колонок сумуються --- це різновид дублювання даних і такі
  рядки треба додатково вичищати
\end{itemize}

На перший погляд маємо парадокс: набір даних опублікований у
структурованому форматі .xlsx і тому є машиночитаним, але для прочитання
в середовищі розробки (машинної обробки) потребує суттєвої участі людини
і тому не є машиночитаним. Цей парадокс зникає, коли детальніше дивимося
на значення, в яких машиночитаність може використовуватись:

\begin{enumerate}
\def\labelenumi{\arabic{enumi}.}
\tightlist
\item
  Машиночитаність як ступінь інтегрованості набору даних у мережу
  Інтернет\footnote{\url{https://blog.ldodds.com/2015/02/20/comparing-the-5-star-scheme-with-open-data-certificates/}}.
  Сюди відносяться вимоги та рекомендації, які дозволяють веб-ресурсам
  правильно і повно взаємодіяти один з одним. Публікація набору даних у
  структурованому форматі .xlsx --- це вже приклад відповідності таким
  вимогам\footnote{\url{https://www.w3.org/2011/gld/wiki/5_Star_Linked_Data}}.
  Це технічна сторона вимог до оформлення веб-ресурсу --- концепція
  \emph{5-Star Linked Data} Тіма Бернерса-Лі є найвідомішим прикладом
  рекомендацій щодо машиночитаності.
\item
  Машиночитаність як чистота даних та відповідність використовуваних
  форматів потребам користувачів. Поняття tidy data, критерії Open Data
  certificate, важливість юз-кейсів/ розуміння, хто буде користуватися
  наборами.
\end{enumerate}

\section{Машиночитаність як інтегрованість набору даних у мережу
Інтернет}\label{ux43cux430ux448ux438ux43dux43eux447ux438ux442ux430ux43dux456ux441ux442ux44c-ux44fux43a-ux456ux43dux442ux435ux433ux440ux43eux432ux430ux43dux456ux441ux442ux44c-ux43dux430ux431ux43eux440ux443-ux434ux430ux43dux438ux445-ux443-ux43cux435ux440ux435ux436ux443-ux456ux43dux442ux435ux440ux43dux435ux442}

\subsection{а. 5-Star Linked}\label{ux430.-5-star-linked}

\section{Машиночитаність як відповідність потребам
користувачів}\label{sec-data-methods}




\end{document}
